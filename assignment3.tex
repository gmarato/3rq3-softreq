\documentclass{article}
\usepackage[margin=1in]{geometry}
\usepackage{enumerate}
\title{Assignment 3}
\author{Marato Gebremichael}
\date{Due 27/10/2020}
\begin{document}
   \maketitle
   
\section*{Problem 1}   

\begin{enumerate}[(a)]

	\item
   
	\begin{itemize}
		\item Hotel manager
		
			\begin{enumerate}
				\item[--] In charge of overseeing all hotel operations including staff, financial and customer satisfaction
				\item[--] Subject matter expert and ultimate decision maker on day to day operations 
				\item[--] Significant input into such individual projects like the elevator system		
				\item[--] Main objective is to generate income by facilitating and keeping great customer experience 
				\item[--] Highest level of influence of system acceptance, main decision maker on acceptance or consulting the CEO to accept
			\end{enumerate}			

		\item Hotel front desk

			\begin{enumerate}
				\item[--] Very high degree of customer interaction
				\item[--] Limited interaction with the system 
				\item[--] Great understanding on how people flow in and around the hotel, as they direct them
				\item[--] Will have some input on the system design 
				\item[--] Very little influence on system acceptance
				\item[--] Main priority is to keep the costumer happy by assisting them on everything   
			\end{enumerate}

		\item Hotel cleaning staff
			\begin{enumerate}
				\item[--] High degree of interaction with the elevator system
				\item[--] Limited understanding on the needs of the system in terms of operations
				\item[--] Limited input on the system design
				\item[--] Little influence on system acceptance
				\item[--] Main priority is to keep the customer happy by cleaning the rooms    
			\end{enumerate}
	
	\end{itemize}

\item Hotel cleaning staff and the hotel concierge use the system in a similar manner and frequency. 
Only one of them would need to be interviewed as they will have similar knowledge of the system.
They have a general and limited understanding on the system needs even though they interact with it 
on a daily basis. They both will have limited input into the system needs as the manager will have all
the data in terms of usage of the elevator by costumers and his staff. And in the end, they will have little
influence on the acceptance of the system, unless something is severely wrong with the system. 

\end{enumerate}

\section*{Problem 2}

\begin{enumerate}[(a)]
\item 
A weak conflict is between requirement \#1 and \#5. The weak conflict is particular for guests who already have an account with a hotel which likely means the credit card is saved on their profile. In this case it wouldn't be necessary to verify the credit card payment, only a ID check would be sufficient before the guest can check in. Requireing a guest with an existing account with a saved credit card on profile would go against making the check in process "as easy as possible." Guests with existing accounts should have an "express check-in" where only the an ID check would be made.


\item 
A strong conflict is between requirement  \#2 and \#4. Requirement \#4 dictates such that guests can access their existing account only with the hotel from  within the hotel premise (either with the front desk or from secured hotel network). \\ 

Perhaps, the wording on requirement \#4 was wrongly devised, if slightly corrected to the following requirement below the conflict is quicly resolved. \\
 
"4. All guest information must be accessed securely from external sources"

\end{enumerate}

\section*{Problem 3} 

\begin{enumerate}[(a)]
	\item "When there hasn’t been any input to the door, it should be locked" \\
	
	This requirement is not complete and vague. As it is stands when there's no input, the door should be locked. Assuming that the locks have the capability to auto-lock. 
	In a scenario where the user enters the code then stops further inputs, would the door auto-lock? 
	The requirement also does not take into account the status of the door, whether it's opened or closed. \\
	
	It would be a good idea to add or change the requirement such that the door should be locked only after a certain period of time and only if the door is fully closed. 
	 
	\item "Upon locking the door, the light should turn red" \\
	
	This requirement is likewise incomplete and not correct. It specifies the requirement for a transition and not the general status of the door.
	It implies that only during the process of locking the door the light should turn red. Rather than specifying that a locked door should show the light as red.
	It will omit the case where locking the door has failed, ex. the door is opened during a locking sequence, in this case the red light would still turn red as the transistion 
	what changes the light color. \\

	Overall, both requirement \#3 and \#4 have the same issue of specifying an action of the light for the transition rather than the status of the lock of the door. 
	However, a bigger issue with the two requirements along with requirement \#2 is the necessity of the all those requirements. 
	It brings value to the user in terms of easy understanding (usability) wether the doors is closed or opened but at the expense of security and safety. 
	With this light feature, not only authorized users will know of the status of the door but unauthorized users as well. 
	One case that comes to mind is if there's no auto-locking feature, or the auto-lock timing is long or the door lock malfunctioned,
	then an unauthorized user would easily know and potentially enter the hotel room. 
	This is an exercise of risk management and mitigation. Depending on how strongly this feature is needed I would advise against it. 
	If this feature required then a critical requirement would be needed to ensure the doors are auto-locked and there's a system feature to advise
	the security personal that a door has failed to be locked or has been unlocked frequently or kept unlocked for a long period of time.



\end{enumerate}}

   
\end{document}