\documentclass[12pt,a4paper]{article}
\usepackage[margin=25mm]{geometry}

\usepackage{titlesec}
\usepackage[toc,page]{appendix}
\setcounter{secnumdepth}{4}

\titleformat{\paragraph}
{\normalfont\normalsize\bfseries}{\theparagraph}{1em}{}
\titlespacing*{\paragraph}{0pt}{3.25ex plus 1ex minus .2ex}{1.5ex plus .2ex}

\begin{document}

\begin{titlepage}
	\centering
	{\scshape\LARGE McMaster University \par}
	\vspace{2cm}
	{\scshape\Large Project Deliverable \#2 \par}
	\vspace{4cm}
	{\huge\bfseries EMG Capturing Software\par}
	\vspace{2cm}
	{\Large\itshape Marato Gebremichael\par}
	
	\vfill
	Course: 3RQ3 – Software Requirements and Specifications\par
    Instructor: Seat Watson

	\vfill

% Bottom of the page
	{\large \today\par}
\end{titlepage}

\tableofcontents

\newpage

\section{Introduction}

This application will be designed to interface, capture and analyze EMG signals from a portable EMG device.  

\section{Description}

\subsection{System outline}

The application is composed of a graphical user interface that contains a suite of controls to manage and display the capturing of the data from the EMG device.

\subsection{Users}

The end-users of the application will include:

\begin{itemize}
	\item Private customer
	\item Trainer
	\item Nurse
\end{itemize}

\subsection{Owners}

The owners and maintainers of the application will include:

\begin{itemize}
	\item Private customer
	\item Trainer 
	\item Nurse
\end{itemize}

The maintenance including bug fixes will be completed via automatic updates for which the end-user is responsible for.

\section{Constraints}

The application must be available on both Windows and Mac operating systems, and must interface existing EMG devices. 

\section{System features}

\subsection{Data capture}

The application shall be able to capture the EMG signal from the device and save it onto the host of application.

\subsection{Data capture length}

The application shall be able to continuously capture up to 1 hour of data.

\subsubsection{Live data capture}

The application shall capture the live EMG signal form the connected device and display on the graph.

\subsubsection{Noise filtering}

The application shall be able to filter DC to low frequency noise ($<$ 100Hz) and high frequency noise ($>$ 5000Hz). 

\subsubsection{Data filtering}

The following filters shall be made available to be applied on the captured data:

\begin{itemize}
\item average
\item peak
\item RMS
\item low-pass filter
\end{itemize}

\subsubsection{Graphing of Data}

The user shall be able to select the following settings of the graph:

\begin{itemize}
\item Horizontal scale (time measured in seconds)
\item Vertical scale (amplitude measured in millivolts)
\item Grid scale 
\item Adding and removing of x and y markers
\end{itemize}

\subsubsection{Data navigation}

The user shall be able to navigate the captured data along the time axis from start to finish of the captured session.

\subsubsection{Graph saving}

The user shall be able to save the current graph in view to a (.png) image format on the application host.

\subsubsection{Data saving}

The user shall be able to save the captured data in (.csv) format along with information about the application version and EMG device (hardware/firmware version and serial number).

\subsubsection{Data recall}

The user shall be able to recall up to two saved captured data in (.csv) format simultaneously and display it on the graph.

\section{Interface}

\subsection{Connecting to an EMG device}

The application shall allow the user to connect to an existing single channel EMG device.

\subsubsection{Connection type}

The application shall allow the user to connect to an EMG device via:

\begin{itemize}
\item Wired connection
\item Wireless
\end{itemize}

\subsubsection{Retrieving device information}

The application shall allow the user to retrieve EMG device information such as: 

\begin{itemize}
\item device local time
\item battery status
\item firmware version
\item hardware version
\item serial number
\end{itemize}

\subsubsection{Firmware upgrade}

The application shall be able to perform firmware updates on user discretion if the firmware is out of date.

\subsubsection{Date synchronization}

The application shall automatically synchronize the EMG device local time with the local time of the application.

\section{Quality attributes}

\subsection{User Interface}

The application shall have an intuitive interface with support for a touch screen interface.

\subsubsection{Touch intuitive controls}

The controls of the interface shall be sufficiently large such that an user could access it via a touch screen.

\subsubsection{Device connection}

The user shall be able to connect to the EMG device within 1 minute of opening the application.

\subsubsection{Data Graphing}

The user shall be able to change the colors from a selected list of the following graph elements:

\begin{itemize}
\item Background color of the graph
\item Color of the grid
\item Color of the trace (data)
\item Color of the markers
\end{itemize}

\subsubsection{Data display}

The application shall display the captured date from the EMG device with a delay $\leq$ 0.5s

\subsubsection{Data Recall}

The application shall be able to recall saved data (up to 1 hour long) in no less than 1 minute.

\subsubsection{Menu bar}

The application must feature a menu bar from which common menu items will be available such as: help, settings, file open/close etc.

\subsubsection{Status bar}

The application must feature a status bar where information regarding the EMG device such as: connection status and serial number will be displayed.

\subsubsection{Closing the application}

The user shall be alerted if attempting to close the application during a capturing session.

\subsubsection{Out of range data}

The user shall be notified if the captured data falls out of range of the current graph setting.

\subsection{Standards Compliance}

The application must be compliant with IEC 62304:2006 standard.

\newpage
\begin{flushleft}
\textbf{{\Large Appendix}}
\end{flushleft}
\appendix
\section{Background Research}



\section{Scenarios}

\begin{thebibliography}{9}

\bibitem{Biometrics}
  Biometrics Ltd,
  \textit{“Biometrics Analysis software.”},
  Accessed 2020-10-28,
  www.biometricsltd.com/emg-software
  
\bibitem{Laborie}
  Laborie Medical Technologies,
  \textit{“NXT is modern urodynamics”},
  Accessed 2020-10-28,
  www.laborie.com/experiencenxtpro/

\end{thebibliography}



\end{document}